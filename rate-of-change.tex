\section*{Rate of Change}

Suppose that a particle moves along some straight line a certain distance depending on time $t$.
Then the distance $s$ is a function of $t$, which we write $s = f(t)$.

For two values of the time $t_1$ and $t_2$, the quotient
\[\frac{f(t_2) - f(t_1)}{t_2 - t_1}\]
can be regarded as a sort of average speed of the particle. We of course see this pattern as the slope of a line
in analytic geometry. We note that distance divided by time should intuitively yield speed. We regard the limit
\[\lim{t\to t_0} \frac{f(t) - f(t_0)}{t - t_0}\]
as the rate of change of $s$ with respect to $t$. This is the derivative $f^\prime(t)$, which is called the
\textbf{speed}.

Let us denote the speed by $v(t)$, that is, speed is a function of time. Then the speed is given by the derivative
of the distance with respect to time, which is
\[v(t) = \frac{ds}{dt}.\]
The rate of change of the speed is called the \textbf{acceleration}. Thus
\[\frac{dv}{dt} = \frac{d^2s}{dt^2},\]
which is the second derivative of the distance with respect to time. We essentially found the rate of change of the
rate of change of distance relative to time is acceleration.

In general, given a function $y = f(x)$, the derivative $f^\prime(x)$ is interpreted as the \textbf{rate of change
  of $y$ with respect to $x$}.
