\section*{Rate of Change}

Suppose that a particle moves along some straight line a certain distance depending on time $t$.
Then the distance $s$ is a function of $t$, which we write $s = f(t)$.

For two values of the time $t_1$ and $t_2$, the quotient
\[\frac{f(t_2) - f(t_1)}{t_2 - t_1}\]
can be regarded as a sort of average speed of the particle. We of course see this pattern as the slope of a line
in analytic geometry. We note that distance divided by time should intuitively yield speed. We regard the limit
\[\lim{t\to t_0} \frac{f(t) - f(t_0)}{t - t_0}\]
as the rate of change of $s$ with respect to $t$. This is the derivative $f^\prime(t)$, which is called the
\textbf{speed}.

Let us denote the speed by $v(t)$, that is, speed is a function of time. Then the speed is given by the derivative
of the distance with respect to time, which is
\[v(t) = \frac{ds}{dt}.\]
The rate of change of the speed is called the \textbf{acceleration}. Thus
\[\frac{dv}{dt} = \frac{d^2s}{dt^2},\]
which is the second derivative of the distance with respect to time. We essentially found the rate of change of the
rate of change of distance relative to time is acceleration.

In general, given a function $y = f(x)$, the derivative $f^\prime(x)$ is interpreted as the \textbf{rate of change
  of $y$ with respect to $x$}.

\textbf{Example}. What is the rate of change of the area of a circle with respect to its radius, diameter,
circumference?

Let $A(r) = \pi r^2$ be the area of a circle as a function of the radius $r$. The rate of change of $A$
with respect to $r$ is simply
\[\frac{dA}{dr} = \pi \cdot 2r = 2\pi r.\]

For the diameter $D$ and circumference $C$, we must give the formula for $A$ in terms of $D$ and $C$ by substituting
$r$ for some equivalent expression. This is trivial, then we find the rates of change and we are done.

\textbf{Example}. A light shines on top of a lamppost $20$ ft above the ground. A woman $5$ ft tall walks away
from the light. Find the rate at which her shadow is increasing if she walks at the rate of $4$ ft/sec.

We do this by means of similar triangles. Let $a$ be the distance between the woman and the lamppost and let $s$ be
the length of the woman's shadow. We have
\[\frac{20}{5} = \frac{a + s}{s}.\]
