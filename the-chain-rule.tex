\section*{The Chain Rule}

\textbf{Chain rule.} \textit{Let $f$ and $g$ be two functions having derivatives, and such that $f$ is defined
  at all numbers which are values of $g$. Then the composite function $f \circ g$ has a derivative, given by
  the formula}
  \[(f \circ g)^\prime(x) = f^\prime(g(x))g^\prime(x).\]

  \textit{Remarks on the proof.} We distinguish two kinds of numbers $h$. Let $H_1$ be the set of $h$ such that
  $g(x + h) - g(x) \ne 0$, and $H_2$ be the set of $h$ such that $g(x + h) - g(x) = 0$.

  For $h$ in $H_1$, we must show that the limit of the Newton quotient of $f \circ g$ is $f^\prime(u)g^\prime(x)$.
  By definition, we have
  \[\frac{f(g(x + h)) - f(g(x))}{h}.\]
  Put $u = g(x)$, as we have practiced before in the examples, and let $k = g(x + h) - g(x)$. Then we have
  \[\frac{f(g(x) + g(x + h) - g(x)) - f(u)}{h} = \frac{f(u + k) - f(u)}{h}.\]
  We have essentially added $0$ to the input of $f$. Since $k$ is expressed in $h$, we say that $k$ depends on
  $h$ and tends to $0$ as $h$ approaches $0$. Since we are dealing with $h$ in $H_1$, then $k$ is unequal to $0$
  for all small values of $h$. Then we can multiply and divide this quotient by $k$, and obtain
  \[\frac{f(u + k) - f(u)}{k} \frac{k}{h} = \frac{f(u + k) - f(u)}{k} \frac{g(x + h) - g(x)}{h}.\]
  Note that we multiply the Newton quotient by $k/k$.
  As $h$ approaches $0$, then our Newton quotient approaches
  \[f^\prime(u)g^\prime(x).\]

  For $h$ in $H_2$, we show that the limit of the Newton quotient of $f \circ g$ is $0$, and that $0$ is equivalent
  to writing the formula for the chain rule anyway. We assume that we have $g(x + h) - g(x) = 0$ for arbitrarily
  small values of $h$. Then
  \[\lim_{h\to0} \frac{f(g(x + h)) - f(g(x))}{h} = 0,\]
  because $g(x + h) - g(x) = 0$, so $g(x + h) = g(x)$ therefore
  \[f(g(x + h)) - f(g(x)) = f(g(x)) - f(g(x)) = 0.\]
  Since the limit approaches $0$ as $h$ approaches $0$, we can choose any number equal to $0$ to represent this
  limit. We choose $f^\prime(g(x))g^\prime(x)$ to keep the formula constant whether $h$ is in $H_1$ or $H_2$.
