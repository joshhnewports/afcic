\section*{Powers}

\indent\textbf{Theorem 4.1.} \textit{Let $n$ be an integer $\ge 1$ and let $f(x) = x^n$. Then
  \[\frac{df}{dx} = nx^{n-1}.\]}

\indent\textit{Remarks on the proof}. When we have some number $(x + h)^n$, writing each factor yields
\[(x + h)(x + h)\cdots (x + h).\]
If we were to distribute we would get many terms that we do not need to think about. We are able to select which
terms from each factor we wish to distribute to find a particular number. There exists $n$ number of $x$ and we
multiply them by each other, giving us $x^n$.

If we choose $x$ from all but one factor, then the remaining factor has $h$ and we get $hx^{n-1}$.
But we do this for each factor. The idea is that it is not the $h$ from one particular factor,
but it could be the $h$ from any factor.
Since the terms are added when we distribute $(x + h)^n$, then we add the $n$ instances of $hx$, and
get $nhx^{n-1}$.

Now we have the term $x^n$ and the only term $nhx^{n-1}$ having a factor of $h^1$. We conclude that every other
term must choose $h$ from at least two factors. Hence we have
\[(x + h)^n = x^n + nhx^{n-1} + h^2g(x, h),\]
where $g(x,h)$ is some expression involving powers of $x$ and $h$ with numerical coefficients. Of course $h^2$
is factored from the expression.

The rest of the proof follows very naturally using the Newton quotient.

\textbf{Theorem 4.2.} \textit{Let $a$ be any number and let $f(x) = x^a$ (defined for $x > 0$). Then $f(x)$ has
  a derivative, which is}
  \[f^\prime(x) = ax^{a-1}.\]

  We do not prove this until we have more techniques available.
