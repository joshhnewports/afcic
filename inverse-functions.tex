\chapter*{Inverse Functions}
A function $f$ has a many-to-one ($y = x^2$) or one-to-one ($y = x$) relation from its domain to its range. If the converse of the latter statement is true, where there is a one-to-one relation of the range to the domain, then we can define the inverse function
\[x = g(y) =\: \text{the unique number $x$ such that}\: y = f(x).\]
We have the fundamental relation
\[f(g(y)) = y\: \text{and}\: g(f(x)) = x.\]
If a function has a many-to-one relation, then the converse of this relation is one-to-many and thus no inverse function can be defined as this would not be a function.

\textbf{Example.} Consider the function $y = x^2$ defined for $x \ge 0$. For all $x$ in the domain there uniquely corresponds a number $y$, and for all $y$ an $x$ (and this statement is true only for $x \ge 0$). Hence we can define the inverse function.

Had we defined the function for all $x$, then there corresponds for all $y \ne 0$ two numbers $x$ such that $y = x^2$, and thus we could not define the inverse function.

\textbf{Theorem 1.1.} \textit{Let $f(x)$ be a function which is strictly increasing. Then the inverse function exists, and is defined on the set of values of $f$.}

\textit{Proof.} Given a number $y_1$ and a number $x_1$ such that $f(x_1) = y_1$, there cannot be another number $x_2$ such that $f(x_2) = y_1$ unless $x_2 = x_1$, because if $x_2 \ne x_1$ then either $x_2 < x_1$ or $x_2 > x_1$, and since $f$ is strictly increasing it is clear that $f(x_2) \ne y_1$ in either case.

\textbf{Theorem 1.2.} \textit{Let $f$ be a continuous function on the closed interval
  \[a \le x \le b\]
  and assume that $f$ is strictly increasing. Let $f(a) = \alpha$ and $f(b) = \beta$. Then the inverse function is defined on the closed interval $[\alpha, \beta]$.}

\textit{Notes on the proof.} The values of $f$ over $a, b$ correspond to $\alpha, \beta$. For the values of $f$ at $[a, b]$, they must be within $[\alpha, \beta]$ as given by the intermediate value theorem. The interval $[\alpha, \beta]$ is the range of $f$. Theorem 1.1. gives that the inverse function of $f$ is defined on this range $[\alpha, \beta]$, and the proof is complete.

Here is an important result. Let $f(x) = x^n$, where $n$ is a positive integer. We view $f$ as defined only for numbers $x > 0$. Since $f^\prime(x) = nx^{n-1}$, the function is strictly increasing (if $n = 1$ then $nx^{n-1} = 1$, and if $n > 1$ then $nx^{n-1} > 0$, and in both cases $nx^{n-1} > 0$). Hence the inverse function exists and is the $n$-th root function.

\textbf{Example.} Let $f(x) = x^2 + 2x - 3$ for $0 \le x$. We have $f^\prime(x) = 2x + 2$. Thus $f^\prime(x) > 0$ iff $x > -1$ and $f^\prime(x) < 0$ iff $x < -1$. Clearly $f$ is never strictly decreasing because its domain does not contain any numbers $x < -1$. But $f$ is indeed strictly increasing for all numbers it is defined for. The range of $f$ is $[2, \inf)$, and this is the domain of the inverse function as given by Theorem 1.1. 

  \textbf{Example.} Let $f(x) = x/(x+1)$ for $-1 < x$. Then $f^\prime(x) = 1/(x+1)^2$. For $x \ne -1$, the divisor is a square is always positive. Hence $f(x)$ is strictly increasing for all $x$ it is defined for. By Theorem 1.1, $f$ has an inverse function $g$. Now we must find what values $g$ is defined on.
  
  Recall that a hyperbola is the graph of an equation $xy = 1$ or $y = 1/x$. Indeed $f$ is a hyperbola with a vertical asymptote at $-1$. We expect one curve on both sides of the asymptote. Yet $f$ is not defined for those points to the left of the asymptote.
  
  As $x \to \inf, f(x) = 1/(1 + 1/x) \to 1$. As $x \to -1$ from the right, $x + 1 \to \inf$ hence $x/(x + 1) \to -\inf$. Therefore the interval of values of $f$ is $(-\inf, -1)$, and this is the domain of of the inverse function of $f$.

  \textbf{Example.} Let $f(x) = x/(x + 2)$ for $-2 < x$. We have $f^\prime(x) = 2/(x + 2)^2$. For $x \ne -2$, we have $f^\prime(x) > 0$. So $f$ is strictly increasing for all numbers it is defined for. As $x \to \inf, f(x) \to 1$. As $x \to -2$ from the right, $x + 2 \to 0$, hence $x/(x + 2) \to -\inf$. Thus the set of values of $f$ is $(-\inf, 1)$, and this is what the inverse function is defined on.

  \textbf{Example.} Let $f(x) = (x + 1)/(x - 1)$ for $1 < x$. We have $f^\prime(x) = -2/(x - 1)^2$. For $x \ne 1$, we have $f^\prime(x) < 0$. Hence $f$ is strictly decreasing for all numbers it is defined for. As $x \to \inf, f(x) \to 1$. As $x \to 1$ from the right, $f(x) \to \inf$. Thus the set of values of $f$ is $(1, \inf)$, and is what the inverse function is defined on.

  \textbf{Example.} Let $f(x) = 1/x^2$ for $0 < x \le 1$. Then $f^\prime(x) = -2/x^3$. Thus $f(x)$ is strictly decreasing. And we find that the range of $f(x)$ is $[1, \inf)$, and this is the domain of the inverse function of $f$.
    
    \textbf{Example.} Let $f(x) = x + 1/x$ for $0 < x \le 1$. Then $f^\prime(x) = 1 + -1/x^2$ and $f^{\prime\prime}(x) = 2/x^3$. We have that $f'(x) < 0$ for $x \ne 1$. Since $1$ is an endpoint of the interval, $f$ is not differentiable at that point. But this does not contradict $f$ from being strictly monotonic at $1$. As $x \to 0, f(x) \to \inf$, and $f(1) = 2$. Indeed the derivative gave us that $f$ is strictly decreasing, but after finding the limits of the domain we deduce that $f$ is still strictly decreasing at $2$, which the derivative could not give us. We know that there \textit{could} be points that could show that $f$ is not strictly decreasing. There would be a change in direction of the curve, but the second derivative shows there is none. Thus we have no other choice but that $f$ is strictly decreasing. The range of $f$ is $[2, \inf)$ and is the domain of the inverse function of $f$.

      \textbf{Example.} Let $f(x) = 2x/(1 + x^2)$ for $-1 \le x \le 1$. Then
      \[-\frac{2(x^2 - 1)}{(1 + x^2)^2}.\]
      We have $f^\prime(x) > 0$ iff $2(1 - x^2) > 0$ iff $1 > x^2$ and is true for the domain where $x \ne \pm 1$. By inspection, $f(-1) = -1$ and $f(1) = 1$. Indeed $f$ is strictly increasing. Thus the inverse function of $f$ exists and its domain is $[-1, 1]$.
      
      \textbf{Theorem 2.1.} \textit{Let $a, b$ be two numbers, $a < b$. Let $f$ be a function which is differentiable on the interval $a < x < b$ and such that its derivative $f^\prime(x)$ is $> 0$ for all $x$ in this open interval. Then the inverse function $g(y)$ exists and we have}
      \[g^\prime(y) = \frac{1}{f^\prime(x)}.\]

      \textit{Proof.} Take the Newton quotient
      \[\frac{g(y + k) - g(y)}{k}.\]
      By the intermediate value theorem, all numbers of the form $y + k$ for small $k$ can be written as a value of $f$. This means that there exists numbers $x, h$ such that under $f$ they take on $y + k$. Let $x = g(y)$ and $h = g(y + k) - g(y)$. Then $g(y + k) = x + h$. By the relation of inverse functions, we have $y + k = f(x + h)$. Hence $k = f(x + h) - y = f(x + h) - f(x)$. The Newton quotient for $g$ can therefore be written
      \[\frac{g(y + k) - g(y)}{k} = \frac{x + h - x}{f(x + h) - f(x)} = \frac{h}{f(x + h) - f(x)}.\]
      This is the reciprocal of the Newton quotient of $f$. As $h \to 0$, then we get
      \[\frac{1}{f^\prime(x)}\]
      and this is $g^\prime(y)$.

      \textbf{Example.} Let $f(x) = -x^3 + 2x + 1$. Find $g^\prime(2)$. We need to find the $x$ such that $f(x) = 2$. Write
      \[2 = -x^3 + 2x + 1,\]
      which is
      \begin{align*}
        1 &= -x^3 + 2x\\
        &= x(-x^2 + 2)\\
        &= x(2 - x^2).
      \end{align*}
      Fortunately for $x = 1$, we get $1 = 1(2 - 1)$, thus $f(1) = 2$.
      
      Now we must find the inverse function of $f$. We have $f^\prime(x) = -3x^2 + 2$. Then $f^\prime(x) < 0$ iff $x > \sqrt{2/3}, x < -\sqrt{2/3}$ and $f^\prime(x) > 0$ iff $-\sqrt{2/3} < x < \sqrt{2/3}$. We select an interval to be the domain of $f$ for which the inverse function of $f$ will have inputs on. Since we want to find $g^\prime(2)$, and $f(1) = 2$, we need the interval containing 1 so that $2$ is a value of $f$. The interval satsifying this is $x > \sqrt{2/3}$ because one such $x = 1$ is $> \sqrt{2/3}$, and $f$ is strictly decreasing on this interval so the inverse function of $f$ is defined.
      
      Now we need $g^\prime(2)$, and is
      \[g^\prime(2) = \frac{1}{f^\prime(g(2))} = \frac{1}{f^\prime(1)} = \frac{1}{-1} = -1.\]
      
      \textbf{Example.} Let $f(x) = (x-1)(x-2)(x-3)$. Find $g^\prime(6)$. We have $f^\prime(x) = 3x^2 - 12x + 11$. Factoring is difficult so we use the quadratic formula to find the roots of $f^\prime$. These are $x = 2 \pm \sqrt{23/3}$. Thus we have three intervals where $f^\prime$ is either positive or negative. By inspection, we have $f^\prime(x) > 0$ iff $x < 2 - \sqrt{23/3}, x > 2 + \sqrt{23/3}$ and $f^\prime(x) < 0$ iff $2 - \sqrt{23/3} < x < 2 + \sqrt{23/3}$.

      We restrict the sine function to the interval
      \[-\frac{\pi}{2} \le x \le \frac{\pi}{2}.\]
      We have $\sin^\prime x > 0$ in that interval except at $x = \pi/2$ and $x = -\pi/2$. Therefore the function is strictly increasing over $[-\pi/2, \pi/2]$. The inverse function exists, and is called the \textbf{arcsine}.

      Let $f(x) = \sin x$, and $x = \arcsin y$. We have
      \[f(0) = \sin(\arcsin y) = \sin 0 = 0.\]
      Clearly since $x = \arcsin y$ and $x = 0$, then $\arcsin y = 0$. Also, since $\sin 0 = 0$, then $arcsin 0 = 0$. Furthermore, we have some important values
      \[\sin -\frac{\pi}{2} = -1\: \text{and}\: \sin \frac{\pi}{2}.\]
      By Theorem 1.2, the arcsine is defined over $-1 \le y \le 1$.

      \textbf{Example.} We have
      \[\arcsin \frac{1}{\sqrt{2}} = \frac{\pi}{4}\]
      because
      \[\sin \frac{\pi}{4} = \frac{1}{\sqrt{2}}.\]

      \textbf{Example.} We have
      \[\arcsin(\sin \frac{3\pi}{4}) = \frac{\pi}{4}\]
      because $\sin 3\pi/4 = 1/\sqrt{2}$ and $\arcsin 1/\sqrt{2} = \pi/4$.

      \textbf{Arccos.} View the cosine function as being defined only on the interval $[0, \pi]$. Take the derivative of $\cos$ with respect to $x$
      \[\cos^\prime x = -\sin x.\]
      We know that $\sin x > 0$ over $(0, \pi)$. Thus $-\sin < 0$ over $(0, \pi)$. So $\cos^\prime x$ is strictly increasing, its inverse function exists, and its domain is
      \[[\cos \pi, \cos 0] = [-1, 1].\]
      Call it \arccos. The derivative is
      \[\arccos^\prime y = \frac{1}{\cos^\prime x} = \frac{1}{-\sin x}.\]
      We can take the limits of $arccos^\prime$ to find its behavior near the endpoints of its range, and find
      \[\lim_{x\to0^+} \frac{1}{-\sin x} = -\inf,\: \text{and}\: \lim_{x\to\pi^-} \frac{1}{-\sin x} = -\inf.\]
      Thus as $x\to0^+$ or $x\to\pi^-$, the curve becomes nearly vertical. We also have that the curve points downward as the derivative is negative. Here are some special values: $\arccos 0 = \pi/2, \arccos 1 = 0, \arccos -1 = \pi$.

      We can write the derivative of $\arrcos y$ explicitly in terms of $y$. Write
      \[\cos^2 x + \sin^2 x = 1\]
      so
      \[|\sin x| = \sqrt{1 - \cos^2 x}.\]
      Since $\sin x \ge 0$ for the interval on which $x$ is defined on, which is $[0, \pi]$. Thus we need not consider when $\sin x < 0$, so we do not look at the negative value case of the absolute value. So we have
      \[\sin x = \sqrt{1 - \cos^2 x}\]
      and $\cos x = y$, thus
      \[\arccos^\prime y = \frac{1}{-\sqrt{1 - y^2}}.\]

      \textbf{Arcsec.} Let $\sec x = 1/\cos x$. We can select an interval naively. Since $\sec x$ is not defined at $x = \pi/2$ or $x = -\pi/2$, we choose the interval $(-\pi/2, \pi/2)$. We find the derivative
      \begin{align*}
        \sec^\prime x &= \frac{-1}{\cos^2 x} \cos^\prime x\\
        &= \frac{\sin x}{\cos^2 x}\\
        &= \tan x \sec x.
      \end{align*}
      Since $\cos x > 0$ for $-\pi/2 < x < \pi/2$, then $\sec x > 0$. Furthermore, $\sin x < 0$ for $-\pi/2 < x < 0$ and $\sin x > 0$ for $0 < x < \pi/2$. We select the interval $0 < x < \pi/2$ instead to be the domain of $\sec x$. For this interval, we have $\tan x > 0$ and $\sec x > 0$. Thus $\sec^\prime x > 0$. So $\sec x$ is strictly increasing over this interval.

      Let us also inspect the endpoints. We have $\sec^\prime 0 = 0$ and $\sec^\prime \pi/2$ is clearly not defined.
      Hence $\sec$ is strictly increasing over $[0, \pi/2)$. We define the inverse function whose domain is the range of $\sec$ at this interval. We have $\sec 0 = 1$ and $\lim_{x\to(\pi/2)^-} \sec x = \inf$. So the inverse function of $\sec$ has the domain $[1, \inf)$.

          The derivative of this inverse function is
          \begin{align*}
            (\sec^{-1})^\prime (x) &= \frac{1}{\sec^\prime (\sec^{-1} x)}\\
            &= \frac{1}{\sec^\prime y}\\
            &= \frac{1}{\tan y \sec y}\\
            &= \frac{1}{\frac{\sin y}{\cos y} \frac{1}{\cos y}}\\
            &= (\frac{\sin y}{\cos^2 y})^{-1}\\
            &= \frac{\cos^2 y}{\sin y},\\
          \end{align*}

          where $\sec^{-1} x$ is the $y$ such that $\sec y = x$.

          \textbf{Example.} Find the derivative of $\arcsin(x^2 - 1)$ with respect to $x$. Say $y = f(x) = x^2 - 1$. Then we must find the derivative of $\arcsin y$ with respect to $x$. Write
          \begin{align*}
            (\arcsin y)^\prime(x) &= \arcsin^\prime(y) \cdot y^\prime(x)\\
            &= \arcsin^\prime(x^2 - 1) \cdot (x^2 - 1)^\prime(x).
          \end{align*}

          The left factor is
          \[\arcsin^\prime(x^2 - 1) = \frac{1}{\sqrt{1 - (x^2 - 1)^2}}\]
          and the right factor is
          \[(x^2 - 1)^\prime(x) = 2x.\]
          Thus we have
          \[(arcsin(x^2 - 1))^\prime = \frac{2x}{\sqrt{1 - (x^2 - 1)^2}}.\]

          \textbf{Example.} Find the derivative of $2/(\arccos 2x)$ with respect to $x$. This is
          \begin{align*}
            (\frac{2}{\arccos 2x})^\prime(x) &= (2 \cdot \frac{1}{\arccos 2x})^\prime(x)\\
            &= 2(\frac{1}{\arccos 2x})^\prime(x)\\
            &= 2 \cdot \frac{-1}{\arccos^2 2x} \cdot (\arccos 2x)^\prime(x)\\
            &= \frac{-2}{\arccos^2 2x} \cdot \arccos^\prime 2x \cdot (2x)^\prime(x)\\
            &= \frac{-2}{\arccos^2 2x} \cdot \frac{1}{-\sqrt{1 - (2x)^2}} \cdot 2\\
            &= \frac{4}{(\arccos^2 2x)\sqrt{1 - (2x)^2}}.
          \end{align*}
           
          \textbf{Example.} Determine the intervals over which the function arcsin is beding upward, and bending downward.

          Recall $\arcsin^\prime x = 1/sin^\prime(\arcsin x) = 1/\sqrt{1 - x^2}$. The domain of arcsin is $[-1, 1] = [\sin -\pi/2, \sin \pi/2]$. The domain of $\arcsin^\prime$ is $(-1, 1)$ as is clear by $1/\sqrt{1 - x^2}$ where the divisor is $0$ at $\pm 1$.
          Find $\arcsin^{\prime\prime} x$. This is
          \begin{align*}
            (\frac{1}{\sin^\prime(\arcsin x)})^\prime &= \frac{-1}{(\sin^\prime(\arcsin x))^2} \cdot (\sin^\prime(\arcsin x))^\prime\\
            &= \frac{-1}{\cos^2(\arcsin x)} \cdot -\sin(\arcsin x) \cdot \frac{1}{\cos(\arcsin x)}\\
            &= \frac{\sin(\arcsin x)}{\cos^3(\arcsin x)}\\
            &= \frac{x}{\cos^3(\arcsin x)}\\
            &= \frac{x}{\cos^2(\arcsin x)\cos(\arcsin x)}.
          \end{align*}
          Since $\arcsin^{\prime\prime}$ is derived from $\arcsin^\prime$, we assume the domain of $x$ for $\arcsin^\prime$. That is $-1 < x < 1$. The range of $\arcsin^\prime x$ over $(-1, 1)$ is $(-\pi/2, \pi/2)$ and $\cos x$ is positive over this interval. This justifies writing $\cos y = \sqrt{1 - \sin^2 y}$ without the absolute value. So we have
          \[\frac{x}{\cos^2(\arcsin x)\sqrt{1 - \sin^2(\arcsin x)}} = \frac{x}{cos^2(\arcsin x)\sqrt{1 - x^2}}.\]
          For the $\cos^2$ factor we take the previous result and square it to get
          \[\frac{x}{(1 - x^2)\sqrt{1 - x^2}}\]
          hence
          \[\arcsin^{\prime\prime} x = \frac{x}{(1 - x^2)\sqrt{1 - x^2}}.\]  
          The domain of $\arcsin^{\prime\prime}$ is the same as the domain of $\arcsin^\prime$, which is $(-1, 1)$.

          Where $x = 0$ is where we have a root of $\arcsin^{\prime\prime} x$ which is an inflection point of $\arcsin x$. So we have the interval $-1 < x < 0$ and $0 < x < 1$ to inspect.

          For $-1 < x < 0$, we have $x < 0$ and $(1 - x^2)\sqrt{1 - x^2} > 0$ so $\arcsin^{\prime\prime} x > 0$ thus $\arcsin x$ bends downward over this interval.

          For $0 < x < 1$, we have $x > 0$ and $(1 - x^2)\sqrt{1 - x^2} > 0$ so $\arcsin^{\prime\prime} x > 0$ thus $\arcsin x$ bends upward over this interval.
