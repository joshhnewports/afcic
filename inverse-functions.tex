\chapter*{Inverse Functions}
A function $f$ has a many-to-one ($y = x^2$) or one-to-one ($y = x$) relation from its domain to its range. If the converse of the latter statement is true, where there is a one-to-one relation of the range to the domain, then we can define the inverse function
\[x = g(y) =\: \text{the unique number $x$ such that}\: y = f(x).\]
We have the fundamental relation
\[f(g(y)) = y\: \text{and}\: g(f(x)) = x.\]
If a function has a many-to-one relation, then the converse of this relation is one-to-many and thus no inverse function can be defined as this would not be a function.

\textbf{Example.} Consider the function $y = x^2$ defined for $x \ge 0$. For all $x$ in the domain there uniquely corresponds a number $y$, and for all $y$ an $x$ (and this statement is true only for $x \ge 0$). Hence we can define the inverse function.

Had we defined the function for all $x$, then there corresponds for all $y \ne 0$ two numbers $x$ such that $y = x^2$, and thus we could not define the inverse function.
