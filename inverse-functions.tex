\chapter*{Inverse Functions}
A function $f$ has a many-to-one ($y = x^2$) or one-to-one ($y = x$) relation from its domain to its range. If the converse of the latter statement is true, where there is a one-to-one relation of the range to the domain, then we can define the inverse function
\[x = g(y) =\: \text{the unique number $x$ such that}\: y = f(x).\]
We have the fundamental relation
\[f(g(y)) = y\: \text{and}\: g(f(x)) = x.\]
If a function has a many-to-one relation, then the converse of this relation is one-to-many and thus no inverse function can be defined as this would not be a function.

\textbf{Example.} Consider the function $y = x^2$ defined for $x \ge 0$. For all $x$ in the domain there uniquely corresponds a number $y$, and for all $y$ an $x$ (and this statement is true only for $x \ge 0$). Hence we can define the inverse function.

Had we defined the function for all $x$, then there corresponds for all $y \ne 0$ two numbers $x$ such that $y = x^2$, and thus we could not define the inverse function.

\textbf{Theorem 1.1.} \textit{Let $f(x)$ be a function which is strictly increasing. Then the inverse function exists, and is defined on the set of values of $f$.}

\textit{Proof.} Given a number $y_1$ and a number $x_1$ such that $f(x_1) = y_1$, there cannot be another number $x_2$ such that $f(x_2) = y_1$ unless $x_2 = x_1$, because if $x_2 \ne x_1$ then either $x_2 < x_1$ or $x_2 > x_1$, and since $f$ is strictly increasing it is clear that $f(x_2) \ne y_1$ in either case.

\textbf{Theorem 1.2.} \textit{Let $f$ be a continuous function on the closed interval
  \[a \le x \le b\]
  and assume that $f$ is strictly increasing. Let $f(a) = \alpha$ and $f(b) = \beta$. Then the inverse function is defined on the closed interval $[\alpha, \beta]$.}

\textit{Notes on the proof.} The values of $f$ over $a, b$ correspond to $\alpha, \beta$. For the values of $f$ at $[a, b]$, they must be within $[\alpha, \beta]$ as given by the intermediate value theorem. The interval $[\alpha, \beta]$ is the range of $f$. Theorem 1.1. gives that the inverse function of $f$ is defined on this range $[\alpha, \beta]$, and the proof is complete.

\textbf{Example.} Let $f(x) = x^2 + 2x - 3$ for $0 \le x$. We have $f^\prime(x) = 2x + 2$. Thus $f^\prime(x) > 0$ iff $x > -1$ and $f^\prime(x) < 0$ iff $x < -1$. Clearly $f$ is never strictly decreasing because its domain does not contain any numbers $x < -1$. But $f$ is indeed strictly increasing for all numbers it is defined for. The range of $f$ is $[2, \inf)$, and this is the domain of the inverse function as given by Theorem 1.1.
