\chapter*{Inverse Functions}
Let $f$ be a function with domain $A$ and range $B$. For each $x$ in $A$, there is exactly one $y$ in $B$ such that $y = f(x)$. For each $y$ in $B$, there is at least one $x$ in $A$ such that $f(x) = y$. Then we can define a new function $g$ on $B$ such that
\[x = g(y) =\: \text{the unique number $x$ such that}\: y = f(x).\]
We have the fundamental relation
\[f(g(y)) = y\: \text{and}\: g(f(x)) = x.\]

\textbf{Example.} Consider the function $y = x^2$ defined for $x \ge 0$. For all $x$ in the domain there uniquely corresponds a number $y$, and for all $y$ an $x$ (and this statement is true only for $x \ge 0$). Hence we can define the inverse function.

Had we defined the function for all $x$, then there corresponds for all $y \ne 0$ two numbers $x$ such that $y = x^2$, and thus we could not define the inverse function.
