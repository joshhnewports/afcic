\chapter*{Inverse Functions}
A function $f$ has a many-to-one ($y = x^2$) or one-to-one ($y = x$) relation from its domain to its range. If the converse of the latter statement is true, where there is a one-to-one relation of the range to the domain, then we can define the inverse function
\[x = g(y) =\: \text{the unique number $x$ such that}\: y = f(x).\]
We have the fundamental relation
\[f(g(y)) = y\: \text{and}\: g(f(x)) = x.\]
If a function has a many-to-one relation, then the converse of this relation is one-to-many and thus no inverse function can be defined as this would not be a function.

\textbf{Example.} Consider the function $y = x^2$ defined for $x \ge 0$. For all $x$ in the domain there uniquely corresponds a number $y$, and for all $y$ an $x$ (and this statement is true only for $x \ge 0$). Hence we can define the inverse function.

Had we defined the function for all $x$, then there corresponds for all $y \ne 0$ two numbers $x$ such that $y = x^2$, and thus we could not define the inverse function.

\textbf{Theorem 1.1.} \textit{Let $f(x)$ be a function which is strictly increasing. Then the inverse function exists, and is defined on the set of values of $f$.}

\textit{Proof.} Given a number $y_1$ and a number $x_1$ such that $f(x_1) = y_1$, there cannot be another number $x_2$ such that $f(x_2) = y_1$ unless $x_2 = x_1$, because if $x_2 \ne x_1$ then either $x_2 < x_1$ or $x_2 > x_1$, and since $f$ is strictly increasing it is clear that $f(x_2) \ne y_1$ in either case.

\textbf{Theorem 1.2.} \textit{Let $f$ be a continuous function on the closed interval
  \[a \le x \le b\]
  and assume that $f$ is strictly increasing. Let $f(a) = \alpha$ and $f(b) = \beta$. Then the inverse function is defined on the closed interval $[\alpha, \beta]$.}

\textit{Notes on the proof.} The values of $f$ over $a, b$ correspond to $\alpha, \beta$. For the values of $f$ at $[a, b]$, they must be within $[\alpha, \beta]$ as given by the intermediate value theorem. The interval $[\alpha, \beta]$ is the range of $f$. Theorem 1.1. gives that the inverse function of $f$ is defined on this range $[\alpha, \beta]$, and the proof is complete.

Here is an important result. Let $f(x) = x^n$, where $n$ is a positive integer. We view $f$ as defined only for numbers $x > 0$. Since $f^\prime(x) = nx^{n-1}$, the function is strictly increasing (if $n = 1$ then $nx^{n-1} = 1$, and if $n > 1$ then $nx^{n-1} > 0$, and in both cases $nx^{n-1} > 0$). Hence the inverse function exists and is the $n$-th root function.

\textbf{Example.} Let $f(x) = x^2 + 2x - 3$ for $0 \le x$. We have $f^\prime(x) = 2x + 2$. Thus $f^\prime(x) > 0$ iff $x > -1$ and $f^\prime(x) < 0$ iff $x < -1$. Clearly $f$ is never strictly decreasing because its domain does not contain any numbers $x < -1$. But $f$ is indeed strictly increasing for all numbers it is defined for. The range of $f$ is $[2, \inf)$, and this is the domain of the inverse function as given by Theorem 1.1.

  \textbf{Example.} Let $f(x) = x/(x+1)$ for $-1 < x$. Then $f^\prime(x) = 1/(x+1)^2$. For $x \ne -1$, the divisor is a square is always positive. Hence $f(x)$ is strictly increasing for all $x$ it is defined for. By Theorem 1.1, $f$ has an inverse function $g$. Now we must find what values $g$ is defined on.
  
  Recall that a hyperbola is the graph of an equation $xy = 1$ or $y = 1/x$. Indeed $f$ is a hyperbola with a vertical asymptote at $-1$. We expect one curve on both sides of the asymptote. Yet $f$ is not defined for those points to the left of the asymptote.

  As $x \to \inf, f(x) = 1/(1 + 1/x) \to 1$. As $x \to -1$ from the right, $x + 1 \to \inf$ hence $x/(x + 1) \to -\inf$. Therefore the interval of values of $f$ is $(-\inf, -1)$, and this is the domain of of the inverse function of $f$.

  \textbf{Example.} Let $f(x) = x/(x + 2)$ for $-2 < x$. We have $f^\prime(x) = 2/(x + 2)^2$. For $x \ne -2$, we have $f^\prime(x) > 0$. So $f$ is strictly increasing for all numbers it is defined for. As $x \to \inf, f(x) \to 1$. As $x \to -2$ from the right, $x + 2 \to 0$, hence $x/(x + 2) \to -\inf$. Thus the set of values of $f$ is $(-\inf, 1)$, and this is what the inverse function is defined on.

  \textbf{Example.} Let $f(x) = (x + 1)/(x - 1)$ for $1 < x$. We have $f^\prime(x) = -2/(x - 1)^2$. For $x \ne 1$, we have $f^\prime(x) < 0$. Hence $f$ is strictly decreasing for all numbers it is defined for. As $x \to \inf, f(x) \to 1$. As $x \to 1$ from the right, $f(x) \to \inf$. Thus the set of values of $f$ is $(1, \inf)$, and is what the inverse function is defined on.

  \textbf{Example.} Let $f(x) = 1/x^2$ for $0 < x \le 1$. Then $f^\prime(x) = -2/x^3$. Thus $f(x)$ is strictly decreasing. And we find that the range of $f(x)$ is $[1, \inf)$, and this is the domain of the inverse function of $f$.

    \textbf{Example.} Let $f(x) = x + 1/x$ for $0 < x \le 1$. Then $f^\prime(x) = 1 + -1/x^2$ and $f^{\prime\prime}(x) = 2/x^3$. We have that $f'(x) < 0$ for $x \ne 1$. Since $1$ is an endpoint of the interval, $f$ is not differentiable at that point. But this does not contradict $f$ from being strictly monotonic at $1$. As $x \to 0, f(x) \to \inf$, and $f(1) = 2$. Indeed the derivative gave us that $f$ is strictly decreasing, but after finding the limits of the domain we deduce that $f$ is still strictly decreasing at $2$, which the derivative could not give us. We know that there \textit{could} be points that could show that $f$ is not strictly decreasing. There would be a change in direction of the curve, but the second derivative shows there is none. Thus we have no other choice but that $f$ is strictly decreasing. The range of $f$ is $[2, \inf)$ and is the domain of the inverse function of $f$.

      \textbf{Example.} Let $f(x) = 2x/(1 + x^2)$ for $-1 \le x \le 1$. Then
      \[-\frac{2(x^2 - 1)}{(1 + x^2)^2}.\]
      We have $f^\prime(x) \ge 0$ over the domain of $f$. 
