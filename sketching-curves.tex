A quadratic function $f(x) = ax^2 + bx + c$ with $a \ne 0$ has only one critical point, when $f^\prime(x) = 2ax + b = 0$ so when $x = -b/2a$.

When we deal with exponents and logarithms, we shall again meet the problem of comparing the quotient of two expressions which become large.

Know that: large positive times large positive is large positive. Large positive times large negative is large negative. Large negative times large negative is large positive.

We emphasize that $\inf$ and $-\inf$ are not numbers.

It is correct to say that there is no number which is the limit of $Q(x)$ as $x \to \inf$ or as $x \to -\inf$.

\textbf{Example.} Consider a polynomial
\[f(x) = x^3 + 2x - 1.\]
We can write it in the form
\[x^3\(1 + \frac{2}{x^2} - \frac{1}{x^3}\).\]
When $x$ becomes very large, the expression
\[1 + \frac{2}{x^2} - \frac{1}{x^3}\]
approaches 1. We have that $f(x)$ behaves very much like $x^3$ when $x$ is very large.

\textbf{Example.} Find the limit of
\[\frac{2x^3 - x}{x^4 - 1}\]
as $x$ becomes large positive or negative.

Let $Q(x) = (2x^3 - x)/(x^4 - 1)$. To take the limit of $Q(x)$ as $x$ becomes large, we factor the largest power of $x$ from both the numerator and denominator. This is
\[\frac{x^3(2 - 1/x^2)}{x^4(1 - 1/x^4)}.\]
Cancelling yields
\[\frac{1}{x} \frac{2 - 1/x^2}{1 - 1/x^4}.\]
Finding limits of $Q(x)$ amounts to finding limits of this expression. As $x$ goes to infinity, the right factor approaches $2$. The other factor approaches $0$. Hence
\[\lim_{x\to\inf} Q(x) = 0.\]
Furthermore
\[\lim_{x\to\pm\inf} Q(x) = 0.\]

We shall assume the upcoming expressions are functions $Q(x)$.

\textbf{Example.} Find the limit of
\[\frac{\sin x}{x}\]
as $x$ becomes large positive or negative.

Write
\[\sin x \frac{1}{x}.\]
The factor $1/x$ approaches $0$ as $x$ approaches infinity, and $\sin x$ oscillates between $-1$ and $1$ no matter how large $x$ is. Hence
\[\lim_{x\to\pm\inf} Q(x) = 0.\]

\textbf{Example.} Find the limit of
\[\cos x \frac{1}{x}\]
as $x$ becomes large positive or negative.

Write
\[\cos x \frac{1}{x}.\]
It is clear that $\cos x$ oscillates between $-1$ and $1$. But $1/x$ approaches $0$ as $x$ goes to infinity. Hence
\[\lim_{x\to\pm\inf} Q(x) = 0.\]

\textbf{Example.} Find the limit of
\[\frac{\sin 4x}{x^3}\]
as $x$ becomes large positive or negative.

Write
\[\sin 4x \frac{1}{x^3}.\]
The expression $1/x^3$ tends to $0$ as $x$ approaches infinity. Hence
\[\lim_{x\to\pm\inf} Q(x) = 0.\]

\textbf{Example.} Find the limit of
\[x^3 - x + 1\]
as $x$ becomes large positive or negative.

Write
\[x^3(1 - \frac{1}{x^2} + \frac{1}{x^3}).\]
The expression $1 - 1/x^2 + 1/x^3$ approaches $1$ as $x$ becomes large. We see that
\[\lim_{x\to\inf} Q(x) = \inf\: \text{and}\: \lim_{x\to-\inf} Q(x) = -\inf.\]

\textbf{Example.} Describe the behavior of the polynomial
\[-x^3 - x + 1\]
as $x$ becomes large positive and large negative.

Write
\[x^3(-1 - \frac{1}{x^2} + \frac{1}{x^3}).\]
Then
\[\lim_{x\to\inf} Q(x) = -\inf\: \text{and}\: \lim_{x\to-\inf} Q(x) = \inf.\]

\textbf{Example.} Describe the behavior of the polynomial
\[x^4 + 3x^2 + 2\]
as $x$ becomes large positive and large negative.

Write
\[x^4(1 + \frac{3}{x} + \frac{2}{x^4}).\]
Then
\[\lim_{x\to\pm\inf} Q(x) = \inf.\]

\textbf{Example.} Describe the behavior of the polynomial
\[-x^4 + 3x^3 + 2\]
as $x$ becomes large positive and large negative.

Write
\[x^4(-1 + \frac{3}{x} + \frac{2}{x}).\]
Then
\[\lim_{x\to\pm\inf} Q(x) = -\inf.\]

\textbf{Example.} Describe the behavior of the polynomial
\[2x^5 + x^2 - 100\]
as $x$ becomes large positive and large negative.

Write
\[x^5(2 + \frac{1}{x^3} - \frac{100}{x^5}).\]
Then
\[\lim_{x\to\inf} Q(x) = \inf\: \text{and}\: \lim_{x\to-\inf} Q(x) = -\inf.\]

\textbf{Example.} Describe the behavior of the polynomial
\[-3x^5 + x + 1000\]
as $x$ becomes large positive and large negative.

Write
\[x^5(-3 + \frac{1}{x^4} + \frac{1000}{x^5}).\]
Then
\[\lim_{x\to\inf} Q(x) = \inf\: \text{and}\: \lim_{x\to-\inf} Q(x) = \inf.\]

Let $a, b$ be numbers, $a < b$. Let $f$ be a continuous function defined on the interval $[a, b]$. Assume that $f^\prime$ and $f^{\prime\prime}$ exist on the interval $a < x < b$. If the second derivative is positive in the interval $a < x < b$, then the slope of the curve is increasing, and we interpret this as meaning that the curve is bending up. Conversely, if the second derivative is 
negative, then this means the curve is bending down.

\textbf{Example.} Let $f(x) = x^2$. Then $f^{\prime\prime} = 2$. This second derivative is always positive hence the curve bends up.

\textbf{Example.} Let $f(x) = \sin x$. We have $f^{\prime\prime}(x) = -\sin x$, and thus $f^{\prime\prime}(x) > 0$ on the interval $\pi < x < 2\pi$. Hence the curve is bending up on this interval.

\textbf{Example.} Determine the intervals where the curve
\[y = -x^3 + 3x - 5\]
is bending up and bending down.

Let $f(x) = -x^3 + 3x - 5$. THen $f^{\prime\prime}(x) = -6x$. Thus:
\begin{align}
  f^{\prime\prime}(x) > 0 &\Leftrightarrow x < 0,\\
  f^{\prime\prime}(x) < 0 &\Leftrightarrow x > 0.
\end{align}
Hence $f$ is bending up if and only if $x < 0$; and $f$ is bending down if and only if $x > 0$.

\textbf{Second derivative test.} Let $f$ be twice continuously differentiable on an open interval, and suppose that $c$ is a point where
\[f^\prime(c) = 0\: \text{and}\: f^{\prime\prime}(c) > 0.\]
Then $c$ is a local minimum point of $f$. On the other hand, if
\[f^{\prime\prime(c) < 0\]
  then $c$ is a local maximum point of $f$.}

A point where a curve changes its behavior from bending up to down (or vice versa) is called an \textbf{inflection point}.

  
