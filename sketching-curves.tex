A quadratic function $f(x) = ax^2 + bx + c$ with $a \ne 0$ has only one critical point, when $f^\prime(x) = 2ax + b = 0$ so when $x = -b/2a$. Knowing whether the parabola bends up or down tells us whether the critical point is a maximum or minimum, and the value $x = -b/2a$ tells us exactly where this critical point lies. The points where the parabola crosses the $x$-axis are determined by the quadratic formula.

When we deal with exponents and logarithms, we shall again meet the problem of comparing the quotient of two expressions which become large.

Know that: large positive times large positive is large positive. Large positive times large negative is large negative. Large negative times large negative is large positive.

We emphasize that $\inf$ and $-\inf$ are not numbers.

It is correct to say that there is no number which is the limit of $Q(x)$ as $x \to \inf$ or as $x \to -\inf$.

\textbf{Example.} Consider a polynomial
\[f(x) = x^3 + 2x - 1.\]
We can write it in the form
\[x^3\(1 + \frac{2}{x^2} - \frac{1}{x^3}\).\]
When $x$ becomes very large, the expression
\[1 + \frac{2}{x^2} - \frac{1}{x^3}\]
approaches 1. We have that $f(x)$ behaves very much like $x^3$ when $x$ is very large.

\textbf{Example.} Find the limit of
\[\frac{2x^3 - x}{x^4 - 1}\]
as $x$ becomes large positive or negative.

Let $Q(x) = (2x^3 - x)/(x^4 - 1)$. To take the limit of $Q(x)$ as $x$ becomes large, we factor the largest power of $x$ from both the numerator and denominator. This is
\[\frac{x^3(2 - 1/x^2)}{x^4(1 - 1/x^4)}.\]
Cancelling yields
\[\frac{1}{x} \frac{2 - 1/x^2}{1 - 1/x^4}.\]
Finding limits of $Q(x)$ amounts to finding limits of this expression. As $x$ goes to infinity, the right factor approaches $2$. The other factor approaches $0$. Hence
\[\lim_{x\to\inf} Q(x) = 0.\]
Furthermore
\[\lim_{x\to\pm\inf} Q(x) = 0.\]

We shall assume the upcoming expressions are functions $Q(x)$.

\textbf{Example.} Find the limit of
\[\frac{\sin x}{x}\]
as $x$ becomes large positive or negative.

Write
\[\sin x \frac{1}{x}.\]
The factor $1/x$ approaches $0$ as $x$ approaches infinity, and $\sin x$ oscillates between $-1$ and $1$ no matter how large $x$ is. Hence
\[\lim_{x\to\pm\inf} Q(x) = 0.\]

\textbf{Example.} Find the limit of
\[\cos x \frac{1}{x}\]
as $x$ becomes large positive or negative.

Write
\[\cos x \frac{1}{x}.\]
It is clear that $\cos x$ oscillates between $-1$ and $1$. But $1/x$ approaches $0$ as $x$ goes to infinity. Hence
\[\lim_{x\to\pm\inf} Q(x) = 0.\]

\textbf{Example.} Find the limit of
\[\frac{\sin 4x}{x^3}\]
as $x$ becomes large positive or negative.

Write
\[\sin 4x \frac{1}{x^3}.\]
The expression $1/x^3$ tends to $0$ as $x$ approaches infinity. Hence
\[\lim_{x\to\pm\inf} Q(x) = 0.\]

\textbf{Example.} Find the limit of
\[x^3 - x + 1\]
as $x$ becomes large positive or negative.

Write
\[x^3(1 - \frac{1}{x^2} + \frac{1}{x^3}).\]
The expression $1 - 1/x^2 + 1/x^3$ approaches $1$ as $x$ becomes large. We see that
\[\lim_{x\to\inf} Q(x) = \inf\: \text{and}\: \lim_{x\to-\inf} Q(x) = -\inf.\]

\textbf{Example.} Describe the behavior of the polynomial
\[-x^3 - x + 1\]
as $x$ becomes large positive and large negative.

Write
\[x^3(-1 - \frac{1}{x^2} + \frac{1}{x^3}).\]
Then
\[\lim_{x\to\inf} Q(x) = -\inf\: \text{and}\: \lim_{x\to-\inf} Q(x) = \inf.\]

\textbf{Example.} Describe the behavior of the polynomial
\[x^4 + 3x^2 + 2\]
as $x$ becomes large positive and large negative.

Write
\[x^4(1 + \frac{3}{x} + \frac{2}{x^4}).\]
Then
\[\lim_{x\to\pm\inf} Q(x) = \inf.\]

\textbf{Example.} Describe the behavior of the polynomial
\[-x^4 + 3x^3 + 2\]
as $x$ becomes large positive and large negative.

Write
\[x^4(-1 + \frac{3}{x} + \frac{2}{x}).\]
Then
\[\lim_{x\to\pm\inf} Q(x) = -\inf.\]

\textbf{Example.} Describe the behavior of the polynomial
\[2x^5 + x^2 - 100\]
as $x$ becomes large positive and large negative.

Write
\[x^5(2 + \frac{1}{x^3} - \frac{100}{x^5}).\]
Then
\[\lim_{x\to\inf} Q(x) = \inf\: \text{and}\: \lim_{x\to-\inf} Q(x) = -\inf.\]

\textbf{Example.} Describe the behavior of the polynomial
\[-3x^5 + x + 1000\]
as $x$ becomes large positive and large negative.

Write
\[x^5(-3 + \frac{1}{x^4} + \frac{1000}{x^5}).\]
Then
\[\lim_{x\to\inf} Q(x) = \inf\: \text{and}\: \lim_{x\to-\inf} Q(x) = \inf.\]

\textbf{Example.} Describe the behavior of the polynomial
\[10x^6 - x^4\]
as $x$ becomes large positive and large negative.

Write
\[x^6(10 - \frac{1}{x^2}).\]
Then
\[\lim_{x\to\pm\inf} Q(x) = \inf.\]

\textbf{Example.} Describe the behavior of the polynomial
\[-3x^6 + x^3 + 1\]
as $x$ becomes large positive and large negative.

Write
\[x^6(-3 + \frac{1}{x^3} + \frac{1}{x^6}).\]
Then
\[\lim_{x\to\pm\inf} Q(x) = -\inf.\]

Let $a, b$ be numbers, $a < b$. Let $f$ be a continuous function defined on the interval $[a, b]$. Assume that $f^\prime$ and $f^{\prime\prime}$ exist on the interval $a < x < b$. If the second derivative is positive in the interval $a < x < b$, then the slope of the curve is increasing, and we interpret this as meaning that the curve is bending up. Conversely, if the second derivative is 
negative, then this means the curve is bending down.

\textbf{Example.} Let $f(x) = x^2$. Then $f^{\prime\prime} = 2$. This second derivative is always positive hence the curve bends up.

\textbf{Example.} Let $f(x) = \sin x$. We have $f^{\prime\prime}(x) = -\sin x$, and thus $f^{\prime\prime}(x) > 0$ on the interval $\pi < x < 2\pi$. Hence the curve is bending up on this interval.

\textbf{Example.} Determine the intervals where the curve
\[y = -x^3 + 3x - 5\]
is bending up and bending down.

Let $f(x) = -x^3 + 3x - 5$. Then $f^{\prime\prime}(x) = -6x$. Thus:
\begin{align}
  f^{\prime\prime}(x) > 0 &\Leftrightarrow x < 0,\\
  f^{\prime\prime}(x) < 0 &\Leftrightarrow x > 0.
\end{align}
Hence $f$ is bending up if and only if $x < 0$; and $f$ is bending down if and only if $x > 0$.

\textbf{Second derivative test.} Let $f$ be twice continuously differentiable on an open interval, and suppose that $c$ is a point where
\[f^\prime(c) = 0\: \text{and}\: f^{\prime\prime}(c) > 0.\]
Then $c$ is a local minimum point of $f$. On the other hand, if
\[f^{\prime\prime}(c) < 0\]
then $c$ is a local maximum point of $f$.}

A point where a curve changes its behavior from bending up to down (or vice versa) is called an \textbf{inflection point}.

%section 3 exercises

\textbf{Example.} Sketch the graph of the curve
\[2x^3 - x^2 - 3x.\]

Let $f(x) = 2x^3 - x^2 -3x$. Now we can refer to the expression by a function. We analyze the behavior of $f(x)$ as $x$ becomes arbitrarily large, determine the regions of increase and decrease as well as its critical points, and determine how it bends.

To find the behavior of $f(x)$ we factor the largest power of $x$ from the whole expression, and this is $x^3(2 - x^{-1} - 3x^{-2})$. Hence
\[\lim_{x\to\inf} f(x) = \inf\: \text{and}\: \lim_{x\to-\inf} f(x) = -\inf.\]

To determine the regions of increase and decrease and the critical points of $f$, we need the derivative of $f$. Let $f^\prime(x) = 6x^2 -2x - 3$. The critical points of $f$ are determined by the numbers $c$ such that $f^\prime(c) = 0$, and these numbers $c$ are called roots. That is, the roots of $f^\prime$ determine the critical points of $f$. Since $f^\prime$ is a quadratic polynomial we can simply find the roots by the quadratic formula. The roots are $(1 + \sqrt{19})/6$ and $(1 - \sqrt{19})/6$. This bounds the number line into three intervals. The graph of $f^\prime(x)$ is a parabola bending up. Hence
\begin{align}
f^\prime(x) > 0 \Leftrightarrow x < \frac{1 - \sqrt{19}}{6},\\
f^\prime(x) < 0 \Leftrightarrow \frac{1 - \sqrt{19}}{6} < x < \frac{1 + \sqrt{19}}{6},\\
f^\prime(x) > 0 \Leftrightarrow \frac{1 + \sqrt{19}}{6} < x.
\end{align}
Therefore $f$ is strictly increasing, strictly decreasing, and strictly increasing at those respective intervals.

Now we determine how $f$ bends. Let $f^{\prime\prime} = 12x - 2$. Then
\begin{align}
f^{\prime\prime}(x) = 0 \Leftrightarrow x = \frac{1}{6},\\
f^{\prime\prime}(x) < 0 \Leftrightarrow x < \frac{1}{6},\\
f^{\prime\prime}(x) > 0 \Leftrightarrow x > \frac{1}{6}.
\end{align}

\textbf{Example.} Sketch the graph of the curve
\[x^3 - 3x^2 + 6x - 3.\]

The behavior is given by $x^3(1 - 3x^{-1} + 6x^{-2} - 3x^{-3})$. Hence
\[\lim_{x\to\inf} f(x) = \inf\: \text{and}\: \lim_{x\to-\inf} f(x) = -\inf.\]

The regions of increase and decrease of $f$ are determined by $f^\prime(x) = 3x^2 - 6x + 6$. The roots of $f^\prime$ are complex, hence there deos not exist any critical points of $f$. Then $f^\prime(x)$ must be always positive or always negative for all $x$. Without testing either interval, we see that the graph of $f^\prime$ is a parabola bending up. The coefficient $a$, which is $3$, is positive. We deduce that $f^\prime(x)$ is positive for all $x$ and thus $f(x)$ is strictly increasing for all $x$.

Now we determine how $f$ bends. Let $f^{\prime\prime}(x) = 6x - 6$. Then
\begin{align}
f^{\prime\prime} = 0 \Leftrightarrow x = 1,\\
f^{\prime\prime} < 0 \Leftrightarrow x < 1,\\
f^{\prime\prime} > 0 \Leftrightarrow x > 1.
\end{align}
