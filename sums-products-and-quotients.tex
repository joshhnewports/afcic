\section*{Sums, Products, and Quotients}

\textbf{Definition.} A function is said to be \textbf{continuous at a point $x$} if and only if
\[\lim_{h\to0} f(x + h) = f(x).\]
A function is said to be \textbf{continuous} if it is continuous at every point of its domain of definition.

\begin{center}
\textit{Let $f$ be a function having a derivative $f^\prime(x)$ at $x$. Then $f$ is continuous at $x$.}
\end{center}

\textit{Remarks on the proof.}
We note that if a function $f(x)$ is continuous at $x$, then it is continuous at every point of its domain of
definition.
The proposition statement states that $f$ has a derivative $f^\prime(x)$ at $x$, this is equivalent to saying that
$f$ is differentiable. So what we wish to prove is:

\begin{center}
\textit{Let $f$ be a function that is differentiable. Then $f$ is continuous.}
\end{center}

We set the Newton quotient of $f$ equal to itself then multiply by $h$ and get
\[h \frac{f(x + h) - f(x)}{h} = f(x + h) - f(x).\]
As $h$ approaches 0, the left term approaches $0f^\prime$. Thus we have
\[\lim_{h\to0} f(x + h) - f(x) = 0f^\prime(x) = 0.\]
This is another way of stating that
\[\lim_{h\to0} f(x + h) = f(x).\]
By definition, $f$ is continuous.

We now show some computational rules.

\textbf{Constant times a function.} \textit{The derivative of $cf$ is then given by the formula}
\[(cf)^\prime(x) = c \cdot f^\prime(x).\]
In the other notation, this reads
\[\frac{d(cf)}{dx} = c \frac{df}{dx}.\]

\textbf{Sum.} \textit{Let $f(x)$ and $g(x)$ be two functions which have derivatives $f^\prime(x)$ and $g^\prime(x)$,
  respectively. Then the sum $f(x) + g(x)$ has a derivative, and}
\[(f + g)^\prime(x) = f^\prime(x) + g^\prime(x).\]
In the other notation, this reads
\[\frac{d(f + g)}{dx} = \frac{df}{dx} + \frac{dg}{dx}.\]

\textbf{Product.} \textit{Let $f(x)$ and $g(x)$ be two functions having derivatives $f^\prime(x)$ and $g^\prime(x)$.
  Then the product function $f(x)g(x)$ has a derivative, which is given by the formula}
\[(fg)^\prime(x) = f(x)g^\prime(x) + g(x)f^\prime(x).\]

\textbf{Special case with quotients.} \textit{Let $g(x)$ be a function having a derivative $g^\prime(x)$, and such
  that $g(x) \ne 0$. Then the derivative of the quotient $1/g(x)$ exists, and is equal to}
\[\frac{d}{dx} \frac{1}{g(x)} = \frac{-1}{g(x)^2} g^\prime(x).\]

\textbf{Quotient.} \textit{Let $f(x)$ and $g(x)$ be two functions having derivatives $f^\prime(x)$ and $g^\prime(x)$
  respectively, and such that $g(x) \ne 0$. Then the derivative of the quotient $f(x)/g(x)$ exists, and is equal to}
\[\frac{g(x)f^\prime(x) - f(x)g^\prime(x)}{g(x)^2}.\]
