\chapter*{Sine and Cosine}

On wholes and parts. Note that I made this up and the terminology is nonstandard.

We wish to determine the area of a sector $S$ having $\theta$ radians in a disc $D$ of radius $r$. To get a part $P$ of some whole $W$, we note that all of a whole is simply $1 \cdot W$. As $0 \le P \le 1$, we determine portions of $W$ by
\[P \cdot W.\]
The whole of the area of $D$ is $\pi r^2$ so
\[A = \text{part} \cdot \pi r^2.\]
The whole of the radians is $2\pi$. We take a portion of this, which is $\theta$. Then there exists some number $q$ such that
\[\theta = 2\pi q.\]
We see that $q = \theta/2\pi$ and we have our part. The area of $S$ is
\[\frac{\theta}{2\pi} \pi r^2 = \frac{\theta r^2}{2}.\]

Determine the length $L$ of an arc of $\theta$ radians on a circle of radius $r$. The whole of the length is $2\pi r$ and the whole of the radians is $2\pi$. Again we take $\theta/2\pi$ and we get that
\[L = \frac{\theta}{2\pi} 2\pi r = r\theta.\] 

\begin{theorem}
  The functions $\sin x$ and $\cos x$ have derivatives and
  \begin{align*}
    \frac{d(\sin x)}{dx} &= \cos x,\\
    \frac{d(\cos x)}{dx} &= -\sin x.
  \end{align*}
\end{theorem}

\begin{proof}
  The Newton quotient of $\sin x$ is
  \[\frac{\sin(x + h) - \sin x}{h}.\]
  Using the addition formula, the Newton quotient is then
  \[\frac{\sin x \cos h + \cos x \sin h - \sin x}{h}.\]
  We factorize and get
  \[\frac{\cos x \sin h + \sin x(\cos h - 1)}{h}.\]
  We separate our quotient:
  \[\cos x \frac{\sin h}{h} + \sin x \frac{\cos h - 1}{h}.\]
  For now, assume that as $h$ approaches $0$ the limits of $(\sin h)/h$ and $(\cos h - 1)/h)$ are $1$ and $0$ respectively.
  It follows that $\cos x$ remains, proving that
  \[\lim_{h\to0} \frac{\sin(x + h) - \sin x}{h} = \cos x.\]
  
  We can get the derivative of $\cos x$ in the same manner, but we do so in terms of $\sin x$ and the chain rule. We know that $\cos x = \sin(x + \pi/2)$. Let $u = x + \pi/2$. Taking the derivative of both sides with respect to $x$, we get
  \[\frac{d(\cos x)}{dx} = \cos u = \cos \left(x + \frac{\pi}{2} \right) = -\sin x,\]
  and our theorem is proven.
\end{proof}

We prove
\[\lim_{h\to0} \frac{\sin h}{h} = 1.\]
Taking the figure given, we see that the area of the small triangle is lesser than the area of the sector, and that is lesser than the area of the big triangle. We see that
\[\frac{|AB|}{|OB|} = \sin h = s\]
and
\[\frac{|CD|}{|OC|} = \frac{|CD|}{1} = t\]
where we equivalently find that
\[t = \frac{\sin h}{\cos h} = \tan h.\]
We could also have found that $t = \sin h/\cos h$ by similar triangles. We do this now. Take $|DC|/|BA| = |OC|/|OA|$.
This relation is the same as $t/s = 1/\cos h$, and multiplying by $s = \sin h$ we get $t = \sin h/\cos h$.

The area of the small triangle is $(\cos h \sin h)/2$, the area of the big triangle is $t/2 = (1/2)\sin h \cos h$, and the area of the sector is $h/2$. Hence
\[\frac{1}{2} \cos h \sin h < \frac{1}{2} h < \frac{1}{2} \frac{\sin h}{\cos h}.\]
Multiplying by $2$, we get
\[\cos h \sin h < h <\frac{\sin h}{\cos h}.\]
Recall we assumed $h > 0$, and know that we take $h$ to be small. Then it follows that $\sin h > 0$ so we multiply the inequalities by $1/\sin h$ to get
\[\cos h < \frac{h}{\sin h} < \frac{1}{\cos h}.\]
As $h$ approaches $0$, both $\cos h$ and $1/\cos h$ approach $1$. Thus $h/\sin h$ is squeezed between two quantities which approach $1$, and therefore $h/\sin h$ must approach $1$ also. This is given by Property $5$ of limits.

Our desired quotient $(\sin h)/h$ is equal to $1/(h/\sin h)$. Since
\[\lim_{h\to0} \frac{1}{h/\sin h} = \frac{\lim 1}{\lim h/\sin h} = \frac{1}{1} = 1,\]
then
\[\lim_{h\to0} \frac{\sin h}{h} = 1.\]

This was determined for $h > 0$. Suppose that $h < 0$. Write $h = -k$, then
\[\frac{\sin(-k)}{-k} = \frac{-\sin k}{-k} = \frac{\sin k}{k}.\]
As $h$ approaches $0$, so does $k$, and this is reduced to the previous case.

For the limit of $(\cos h -1)/h$, we take
\begin{align*}
  \frac{\cos h - 1}{h} &= \frac{(\cos h -1)(\cos h + 1)}{h(\cos h + 1)}\\
  &= \frac{\cos ^2 h - 1}{h(\cos h + 1)}\\
  &= \frac{-\sin ^2 h}{h(\cos h + 1)}\\
  &= -\frac{\sin h}{h} \sin h \frac{1}{\cos h + 1}.
\end{align*}
Taking the limit of each factor, which are easily determined, we find that
\[\lim_{h\to0} \frac{\cos h -1}{h} = 0.\]
