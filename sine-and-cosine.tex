\chapter*{Sine and Cosine}

On wholes and parts. Note that I made this up and the terminology is nonstandard.

We wish to determine the area of a sector $S$ having $\theta$ radians in a disc $D$ of radius $r$. To get a part $P$ of some whole $W$, we note that all of a whole is simply $1 \cdot W$. As $0 \le P \le 1$, we determine portions of $W$ by
\[P \cdot W.\]
The whole of the area of $D$ is $\pi r^2$ so
\[A = \text{part} \cdot \pi r^2.\]
The whole of the radians is $2\pi$. We take a portion of this, which is $\theta$. Then there exists some number $q$ such that
\[\theta = 2\pi q.\]
We see that $q = \theta/2\pi$ and we have our part. The area of $S$ is
\[\frac{\theta}{2\pi} \pi r^2 = \frac{\theta r^2}{2}.\]

Determine the length $L$ of an arc of $\theta$ radians on a circle of radius $r$. The whole of the length is $2\pi r$ and the whole of the radians is $2\pi$. Again we take $\theta/2\pi$ and we get that
\[L = \frac{\theta}{2\pi} 2\pi r = r\theta.\] 
