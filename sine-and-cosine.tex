\chapter*{Sine and Cosine}

On wholes and parts. Note that I made this up and the terminology is nonstandard.

We wish to determine the area of a sector $S$ having $\theta$ radians in a disc $D$ of radius $r$. To get a part $P$ of some whole $W$, we note that all of a whole is simply $1 \cdot W$. As $0 \le P \le 1$, we determine portions of $W$ by
\[P \cdot W.\]
The whole of the area of $D$ is $\pi r^2$ so
\[A = \text{part} \cdot \pi r^2.\]
The whole of the radians is $2\pi$. We take a portion of this, which is $\theta$. Then there exists some number $q$ such that
\[\theta = 2\pi q.\]
We see that $q = \theta/2\pi$ and we have our part. The area of $S$ is
\[\frac{\theta}{2\pi} \pi r^2 = \frac{\theta r^2}{2}.\]

Determine the length $L$ of an arc of $\theta$ radians on a circle of radius $r$. The whole of the length is $2\pi r$ and the whole of the radians is $2\pi$. Again we take $\theta/2\pi$ and we get that
\[L = \frac{\theta}{2\pi} 2\pi r = r\theta.\] 

\begin{theorem}
  The functions $\sin x$ and $\cos x$ have derivatives and
  \begin{align*}
    \frac{d(\sin x)}{dx} &= \cos x,\\
    \frac{d(\cos x)}{dx} &= -\sin x.
  \end{align*}
\end{theorem}

\begin{proof}
  The Newton quotient of $\sin x$ is
  \[\frac{\sin(x + h) - \sin x}{h}.\]
  Using the addition formula, the Newton quotient is then
  \[\frac{\sin x \cos h + \cos x \sin h - \sin x}{h}.\]
  We factorize and get
  \[\frac{\cos x \sin h + \sin x(\cos h - 1)}{h}.\]
  We separate our quotient:
  \[\cos x \frac{\sin h}{h} + \sin x \frac{\cos h - 1}{h}.\]
  For now, assume that as $h$ approaches $0$ the limits of $(\sin h)/h$ and $(\cos h - 1)/h)$ are $1$ and $0$ respectively.
  It follows that $\cos x$ remains, proving that
  \[\lim_{h\to0} \frac{\sin(x + h) - \sin x}{h} = \cos x.\]
  
  We can get the derivative of $\cos x$ in the same manner, but we do so in terms of $\sin x$ and the chain rule. We know that $\cos x = \sin(x + \pi/2)$. Let $u = x + \pi/2$. Taking the derivative of both sides with respect to $x$, we get
  \[\frac{d(\cos x)}{dx} = \cos u = \cos \left(x + \frac{\pi}{2} \right) = -\sin x,\]
  and our theorem is proven.
\end{proof}
