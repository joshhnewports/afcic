\chapter*{The Mean Value Theorem}

\begin{definition}
  Let $f$ be a differentiable function. A \textbf{critical point} of $f$ is a number $c$ such that
  \[f^\prime(c) = 0.\]
\end{definition}

\begin{definition}
  We shall say that $c$ is a \textbf{maximum point} of the function $f$ if and only if $f(c) \ge f(x)$ for all numbers $x$ at which $f$ is defined. If the condition $f(c) \ge f(x)$ holds for all numbers $x$ in some interval, then we say that the function has a maximum at $c$ in that interval. We call $f(c)$ a maximum value.
\end{definition}

\begin{definition}
  A \textbf{minimum point} for $f$ is a number $c$ such that $f(c) \le f(x)$ for all $x$ where $f$ is defined. A minimum value for the function is the value $f(c)$, taken at a minimum point.
\end{definition}

We see that the definitions for maximum and minimum points are analogous. Similarly, the next two definitions are analogous.

We shall say that a point $c$ is a \textbf{local minimum} or \textbf{relative minimum} of the function $f$ if there exists an interval $a_1 < c < b_1$ such that $f(c) \le f(x)$ for all numbers $x$ with $a_1 \le x \le b_1$.
